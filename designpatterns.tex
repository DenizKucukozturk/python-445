\documentclass[trans,compress,xcolor=table]{beamer}
%\documentclass{article}
%\usepackage{beamerarticle}
\usepackage{pstricks,pst-node,pst-uml} % PSTricks package
\usepackage[utf8]{inputenc}
\usepackage{listings}
\usepackage{multicol}
%\includeonlyframes{current}

\def\circtxt#1{$\mathalpha \bigcirc \mkern-13mu \mathtt #1$}
\def\colorline#1{\cr \noalign{\color{#1} \hrule height 1pt \vskip-3em}}
\def\colorfline#1{\noalign{\color{#1} \hrule height 1pt}}

\mode<article>
{
  \usepackage{fullpage}
  \usepackage{pgf}
  \usepackage{hyperref}
}

\mode<presentation>
{
  \usetheme{Antibes}
  \usecolortheme[rgb={0.9,0.4,0.1}]{structure}

  \setbeamercovered{transparent}
}


\title{Sofware Development with Scripting Languages:\\Design Patterns}
\author{Onur Tolga Şehitoğlu}
\institute{Computer Engineering,METU}
\subject{Design Patterns}
\date{2 April 2012}

\begin{document}
\lstset{language=Python,
        basicstyle=\scriptsize\ttfamily,
        keywordstyle=\color{blue!50!black}\bfseries,
        identifierstyle=\color{blue!60!green}\sffamily,
        stringstyle=\color{red!70!green}\ttfamily,
	commentstyle=\color{blue!30!white}\itshape,
        showstringspaces=true}

\setbeamercolor{pexample}{bg=orange!5!white,fg=black}%


 \frame{\maketitle}

 \frame{\tableofcontents}

\section{Motivation}
\begin{frame}
\frametitle{Motivation}
\begin{itemize}
\item Object Oriented design $\leftarrow$ code reusability.
\item Solving similar problems in similar ways.
\item Can we re-use design as well?
\item Design patterns are \structure{not} algorithms, data structures, libraries.
\item Reusable solution to a commonly occuring problem.
\item Origins: architecture. Applied to software development later.
\end{itemize}
\end{frame}

\section{Classification of Design Patterns}
\begin{frame}
\frametitle{Classification of Design Patterns based}
\begin{itemize}
\item Creational: creation of new objects
\item Structural: relationships between entities
\item Behavioural: communication patterns between objects
\item Domain specific: (user interface, visualization, security, server,
	concurrency, ...)
\end{itemize}
\end{frame}

\section{Creational Patterns}
\subsection{Singleton}
\begin{frame}
\frametitle{Singleton Pattern}
\begin{itemize}
\item \structure{Problem:} You need only one instance of an object in the application
\item For example database, file store, game board, interface object, etc.
\item \structure{Naive solution:} Use global a variable, just don't
create the second object. You are the programmer aren't you?
\item \structure{Problem with the naive solution:} Global namespace gets 
crowded variable name vs class name confusion, you need to make
sure second object is not created, always need to create the object
even not used.
\item \structure{Singleton:} A special class implementation that
returns always the same object when created.
\end{itemize}
\end{frame}

\defverbatim[colored]\codesinga{
\begin{lstlisting}[escapeinside=\{\}]
class Singleton(object):
    def __new__(cls, *a, **k):
        if not hasattr(cls, '_inst'):
	    # super calls __new__ of the 'cls'
            cls._inst = super(Singleton, cls).__new__(cls, *a, **k)
	return cls._inst
class Counter(Singleton):
    def __init__(self):
	if not hasattr(self,'val'):
		self.val = 0
    def get(self):
	return self.val
    def incr(self):
	self.val += 1

a = Counter()
b = Counter()
Counter().incr()
Counter().get()
\end{lstlisting}
}
\begin{frame}
\begin{beamercolorbox}{pexample}
\codesinga
\end{beamercolorbox}
\end{frame}

\defverbatim[colored]\codesingb{
\begin{lstlisting}[escapeinside=\{\}]
def singleton(cls):
   _instances = {}   # keep classname vs. instance
   def getinstance():
      '''if cls is not in _inst.. create, store, return the instance'''
      if cls not in _instances:
         _instances[cls] = cls()
      return _instances[cls]
   return getinstance

@singleton
class Counter(object):
    def __init__(self):
	if not hasattr(self,'val'): self.val = 0
    def get(self):
	return self.val
    def incr(self):
	self.val += 1

a = Counter()
Counter().incr()
Counter().get()
\end{lstlisting}
}
\begin{frame}
Subclassing other superclasses are problems. A python decorator
version:
\begin{beamercolorbox}{pexample}
\codesingb
\end{beamercolorbox}
\end{frame}

\defverbatim[colored]\codepydec{
\begin{lstlisting}[escapeinside=\{\}]
def dec():
    def f():
      .....
    return f

# ---
@dec
class cls(): {class defn here}
# equivalent to 
class cls(): {class defn here}
cls = dec( cls )
# ---

obj = cls()     # call actually made on f() inside dec()
\end{lstlisting}
}
\begin{frame}
\begin{itemize}
\item Python decorators are evaluated during function/class definition
	and maps function/class definition into a new one
\item Decorators are functions returning a callable that replaces
	function/class constructor
\begin{beamercolorbox}{pexample}
\codepydec
\end{beamercolorbox}
\end{itemize}
\end{frame}

\subsection{Factory Pattern}
\begin{frame}
\frametitle{Factory Pattern}
\begin{itemize}
\item \structure{Problem:} Create objects from a set of classes. Implementation
	depends on class definitions. Introduction of new classes and 
	other changes need recompilation. Class constructors exposed.
\item \structure{Solution:} Use interface functions/objects to encapsulate
	class names and constructors.
	Use a interface functions, methods to provide you instances. Rest is
	handled by polymorphism.
\item Central lifetime management and recycling of objects are possible.
\end{itemize}
\end{frame}

\defverbatim[colored]\codefactory{
\begin{lstlisting}[basicstyle=\tiny\ttfamily,escapeinside=\{\}]
class Player(object):
    def getName(self):
        return self.name
    def hitpower(self):
        return self.hit

class Peasant(Player):
   def __init__(self, name):
       self.name = name
       self.hit = 1
class Warior(Player):
   def __init__(self,name):
      self.name = name
      self.hit = 10

@singleton
class PlayerFactory(object):
   def new(self,name,type):
       if type == 'peasant':
	    return Peasant(name)
       elif type == 'warior':
 	    return Warior(name)
       else:
	    return None

a = PlayerFactory().new('ali','peasant')
b = PlayerFactory().new('veli','warior')
\end{lstlisting}
}
\begin{frame}
\begin{beamercolorbox}{pexample}
\codefactory
\end{beamercolorbox}
\end{frame}

\section{Structural Patterns}
\subsection{Facade}
\begin{frame}
\frametitle{Facade Pattern}
\begin{itemize}
\item \structure{Problem:} A complex library, tool or system;
	consisting of many functions and/or classes; probably
	poorly designed
	is tried to be accessed. It is hard to read and understand. 
	There are many dependencies distributed in the source,  needs
	many housekeeping tasks and stages to access. Any change
	in the system require changes in the whole source code.
\item \structure{Solution:}
	Define a class implementing all details of the 
	library/system and providing a simple uniform interface. Access
	the library through this interface.
\end{itemize}
\end{frame}

\defverbatim[colored]\codefacade{
\begin{lstlisting}[escapeinside=\{\}]
'''libraries here'''
class AuthFacade(object):
   def __init__(self,method):
       if method == 'passwd':
	    # ... use crypt for linux password authentication
       elif method == 'oauth':
 	    # ... use oauth methods to check authentication
       elif method == 'otp':
 	    # ... use one time password routinges for authentication
       else:
	    throw ...
   def auth(self, identity, data):
	''' check authentication based on setting'''

a=AuthFacade('otp')
result  = a.auth('onur','123456')   
\end{lstlisting}
}
\begin{frame}
\begin{beamercolorbox}{pexample}
\codefacade
\end{beamercolorbox}
\end{frame}

\subsection{Proxy Pattern}
\begin{frame}
\frametitle{Proxy Pattern}
\begin{itemize}
\item \structure{Problem:} You need to access a hard to duplicate, 
	limited, probably old class definition. You donot have
	a chance to improve it or change the interface. Or, you want
	to have restricted access to methods (authorization). Or,
	you want smarter access like caching.
\item \structure{Solution:} Write a class interface implementing
	functionalities missing in the original interface. 
\end{itemize}
\end{frame}

\defverbatim[colored]\codeproxy{
\begin{lstlisting}[escapeinside=\{\}]
@singleton
class Counter(object):
   def __init__(self):
	self.val = 0
   def incr(self):
	self.val += 1
   def get(self):
	return self.val

class CounterProxy(object):
   def __init__(self, type):
	self.type = type
   def incr(self):
	if type == 'W':
		Counter().incr()
   def get(self):
	return Counter().get()

a = CounterProxy('R')
b = CounterProxy('W')
a.incr()
b.incr()
\end{lstlisting}
}
\begin{frame}
\begin{beamercolorbox}{pexample}
\codeproxy
\end{beamercolorbox}
\end{frame}

\subsection{Proxy}
\frametitle{Proxy, Facade and Adapter}
\begin{frame}
\begin{itemize}
\item \structure{Proxy}, \structure{Facade}, and \structure{Adapter} provide similar
	encapsulation.
\item In \structure{Proxy}, interface of the wrapped class is same with the \structure{Proxy}. It adds extra functionality (i.e. accounting, caching, remote object, access control)
\item In \structure{Adapter}, interface of the wrapped class is different than \structure{Adapter} class. 
Caller interface is mapped to the adapted interface.
\item In \structure{Facade}, there is no single class to be wrapped but a set of classes, a subsystem. \structure{Facade} simplifies it.
\end{itemize}
\end{frame}

\subsection{Decorator}
\begin{frame}
\frametitle{Decorator}
\begin{itemize}
\item \structure{Problem:} Objects having different behaviour for different 
run-time settings. 
\item \structure{Naive solution:} Have a subclass for each different
combination of attributes. $n$ attributes $\rightarrow$ $2^n$ subclasses.
\item \structure{Solution:} Create a decorator class wrapping the original
behaviour. Cascading behaviours for each new decorator.
\end{itemize}
\end{frame}

\defverbatim[colored]\codedecor{
\begin{lstlisting}[escapeinside=\{\}]
class Coffee(object):
   def __init__(self,size):
	self.size = size
   def cost(self):
	if self.size == 'S': return 2.0
	elif self.size == 'M': return 3.0
	elif self.size == 'L': return 3.5
	else raise InvalidSize

class CoffeeDecorator(object):
        def __init__(self,w):
                self._w = w  #wrapped object
        def __getattr__(self,name):
                return getattr(self._w, name)
class WithMilk(CoffeeDecorator):
	def cost(self):
		return 1.0 + self._w.cost()
class WithCaramel(CoffeeDecorator):
	def cost(self):
		return 0.5 + self._w.cost()

a = WithCaramel(WithMilk(Coffee('L')))
print a.cost()
\end{lstlisting}
}
\begin{frame}
\begin{beamercolorbox}{pexample}
\codedecor
\end{beamercolorbox}
\end{frame}

\section{Behavioral Patterns}
\subsection{Observer}
\begin{frame}
\frametitle{Observer}
\begin{itemize}
\item \structure{Problem:} Objects depending on eachothers states need to
be informed when the other encountered a change. Event handling systems.
\item \structure{Solution:} Maintain a registry of observing objects
in Subject object. When an event occurs, \structure{notify} the observers.
\end{itemize}
\end{frame}

\defverbatim[colored]\codeobs{
\begin{lstlisting}[escapeinside=\{\},basicstyle=\tiny\ttfamily]
class Subject(object):
	_observers = []
	def register(self,obs):
		self._observers.append(obs)
	def unregister(self,obs):
		self._observers.remove(obs)
	def notify(self):
		for o in self._observers:
			o.update(self)
	def state(self): pass

class Observer(object):
	def update(self,subj): pass

class Clock(Subject):
	def __init__(self):
		self.value = 0
	def state(self):
		return self.value
	def tick(self):
		self.value += 1
		self.notify()

class Person(Observer):
	def update(self, obj):
		print "heyo ",obj.state()

a = Person()
b = Clock()
b.register(a)
b.tick()
\end{lstlisting}
}
\begin{frame}
\begin{beamercolorbox}{pexample}
\codeobs
\end{beamercolorbox}
\end{frame}

\subsection{Strategy}
\begin{frame}
\frametitle{Strategy}
\begin{itemize}
\item \structure{Problem:} Combination of behaviours  from multiple class 
hierarchies.
\item \structure{Naive Solution:} Use multiple inheritence. However for
each combination of classes a new derivation required. Adding new behaviours
is not easy.
\item \structure{Solution:} Encapsulate behavours using interfaces and
keep references in the actual class.
\end{itemize}
\end{frame}

\subsection{State}
\begin{frame}
\frametitle{State}
\begin{itemize}
\item \structure{Problem:} A class needs to behave differently based
on its current state. 
\item \structure{Naive solution:} Keep state in member variables
and use conditional statements to implement specific behaviours
at each method. Each method depends on state and conditions replicated.
Changing a state need update of conditionals at each method.
\item \structure{Solution:} Encapsulate state and implement each
	behaviour on concrete classes of \structure{State} depending
	on current state.
\end{itemize}
\end{frame}

\defverbatim[colored]\codestate{
\begin{lstlisting}[escapeinside=\{\},basicstyle=\tiny\ttfamily]
class Cursor(object):
	def __init__(self):
		self.current = Select()
		mouse = "up"
		start = None
	def mouseDown(self,point):
		self.current.mouseDown(self,point)
	def mouseUp(self):
		self.current.mouseUp(self,point)
	def setTool(self,tool):
		self.current = tool
class State(object):
	def mouseDown(self, obj, point):
		self.obj.start = point
class Select(State):
	def mouseUp(self, cur, point):
		if figure.selected:
			figure.selected.move(cur.start, point)
		else
			figure.selectobjects(cur.start, point)
class Line(State):
	def mouseUp(self, cur, point):
		figure.newlineobject(self.cur.start, point)
class Erase(State):
	def mouseUp(self, cur, point):
		figure.deleteobjects(cur.start, point)

a = Cursor()
a.mouseDown((10,0))
a.mouseUp((0,20))
a.setTool = Line()
a.mouseDown((10,10))
a.mouseUp((20,20))
\end{lstlisting}
}
\begin{frame}
\begin{beamercolorbox}{pexample}
\codestate
\end{beamercolorbox}
\end{frame}

\subsection{Command}
\begin{frame}
\frametitle{Command}
\begin{itemize}
\item \structure{Problem:} Client program want to send commands but
does not want to know internals of how command is executing.
Providing extra functionalities like undo, macro, transaction,
composition etc.
\item \structure{Solution:} Encapsulate information for a method call
	inside of a command object.
	(object, method and parameters)
\item Client, invoker and receiver. Invoker keeps track of method calls
	and functionalities.
\item Muli-level undo (stack of method calls), transatcions, progress
	bars, wizards, macro records, parallel processing.
\end{itemize}
\end{frame}

\section{Architectural Patterns}
\subsection{Model View Controller}
\begin{frame}
\frametitle{Model View Controller}
\begin{itemize}
\item \structure{Architectural Patterns} patterns of software architecture
\item \structure{Model View Controller}: Split of roles in an
	application.
	\begin{itemize}
	\item  Model is the business logic and internal representation.
	\item View is the user interface,  presentation of information.
	\item Controller is the user input.
	\end{itemize}
\item Interactions
	\begin{itemize}
	\item user interaction $\rightarrow$ handler in the controller
	\item controller $\rightarrow$ model update, change
	\item model change $\rightarrow$ view  (observer) or \\
		controller $\rightarrow$ view  inform about the change
	\item view $\rightarrow$ model, load the changes and render
		the new information.
	\end{itemize}
\end{itemize}
\end{frame}


\end{document}
